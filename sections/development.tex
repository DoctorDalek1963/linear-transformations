% lintrans - The linear transformation visualizer
% Copyright (C) 2021-2022 D. Dyson (DoctorDalek1963)

% This program is licensed under GNU GPLv3, available here:
% <https://www.gnu.org/licenses/gpl-3.0.html>

\documentclass[../main.tex]{subfiles}

\begin{document}

Please note, throughout this section, every code snippet will have two comments at the top. The first is the git commit hash that the snippet was taken from\footnote{A history of all commits can be found in the GitHub repository\cite{lintrans-github}}. The second comment is the file name and line numbers. If the line numbers are omitted, then the snippet is the whole file. After a certain point, I introduced copyright comments at the top of every file. These are always omitted here.

\subsection{Matrices backend\label{subsection:matrices-backend}}

\subsubsection{\texttt{MatrixWrapper} class}

The first real part of development was creating the \texttt{MatrixWrapper} class. It needs a simple instance dictionary to be created in the constructor, and it needs a way of accessing the matrices. I decided to use Python's \texttt{\_\_getitem\_\_} and \texttt{\_\_setitem\_\_} special methods\cite{python-3-special-methods} to allow indexing into a \texttt{MatrixWrapper} object like \texttt{wrapper['M']}. This simplifies using the class.
%: 29ec1fedbf307e3b7ca731c4a381535fec899b0b
%: src/lintrans/matrices/wrapper.py

This code is very simple. The constructor (\texttt{\_\_init\_\_}) creates a dictionary of matrices which all start out as having no value, except the identity matrix \textbf{I}. The \texttt{\_\_getitem\_\_} and \texttt{\_\_setitem\_\_} methods allow the user to easily get and set matrices just like a dictionary, and \texttt{\_\_setitem\_\_} will raise an error if the name is invalid. This is a very early prototype, so it doesn't validate the type of whatever the user is trying to assign it to yet. This validation will come later.

I could make this class subclass \texttt{dict}, since it's basically just a dictionary at this point, but I want to extend it with much more functionality later, so I chose to handle the dictionary stuff myself.

I then had to write unit tests for this class, and I chose to do all my unit tests using a framework called \texttt{pytest}.
%: 29ec1fedbf307e3b7ca731c4a381535fec899b0b
%: tests/test_matrix_wrapper.py

These tests are quite simple and just ensure that the expected behaviour works the way it should, and that the correct errors are raised when they should be. It verifies that matrices can be assigned, that every valid name works, and that the identity matrix \textbf{I} cannot be assigned to.

The function decorated with \mintinline{python}{@pytest.fixture} allows functions to use a parameter called \texttt{wrapper} and \texttt{pytest} will automatically call this function and pass it as that parameter. It just saves on code repetition.

\end{document}
