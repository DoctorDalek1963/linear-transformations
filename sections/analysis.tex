\documentclass[../main.tex]{subfiles}

\begin{document}

One of the topics in the A Level Further Maths course is linear transformations, as represented by matrices. This is a topic all about how vectors move and get transformed in the plane. It's a topic that lends itself exceedingly well to visualization, but students often find it hard to visualize this themselves, and there is a considerable lack of good tools to provide visual intuition on the subject. There is the YouTube series \textit{Essence of Linear Algebra} by 3blue1brown\cite{essence-of-linear-algebra}, which is excellent, but I couldn't find any good interactive visualizations.

My solution is to develop a desktop application that will allow the user to define 2x2 matrices and view these matrices and compositions thereof as linear transformations of a 2D plane. This will give students a way to get to grips with linear transformations in a more hands-on way, and will give teachers the ability to easily and visually show concepts like determinants and invariant lines.

\subsection{Computational Approach}

This solution is particularly well suited to a computational approach since it is entirely focussed on visualizing transformations, which require complex mathematics to properly display. It will also have lots of settings to allow the user to configure aspects of the visualization. As previously mentioned, visualizing transformations in one's own head is difficult, so a piece of software to do it would be very valuable to teachers and learners, but current solutions are considerably lacking.

My solution will make use of abstraction by allowing the user to define a set of matrices which they can use in expressions. This allows them to use a matrix multiple times and they don't have to keep track of any of the numbers. All the actual processing and mathematics happens behind the scenes and the user never has to worry about it; they just compose their defined matrices into transformations. This abstraction allows the user to focus on exploring the transformations themselves without having to do any actual computations. This will make learning the subject much easier, as they will able to gain a visual intuition for linear transformations without worrying about computation until after they've built up that intuition.

I will also employ decomposition and modularization by breaking the project down into many smaller parts, such as one module to keep track of defined matrices, one module to validate and parse matrix expressions, one module for the main GUI, as well as sub-modules for the widgets and dialog boxes, etc. This decomposition allows for simpler project design, easier code maintenance (since module coupling is kept to a minimum, so bugs are isolated in their modules), inheritance of classes to reduce code repetition, and unit testing to inform development. I also intend this unit testing to be automated using GitHub Actions.

Selection will also be used widely in the application. The GUI will provide many settings for visualization, and these settings will need to be checked when rendering the transformation. For example, the user will have the option to render the determinant, so I will need to check this setting on every render cycle and only render the determinant parallelogram if the user has enabled that option. The app will have many options for visualization, which will be useful in learning, but if all these options were being rendered at the same time, then there would be too much information for the user to properly process, so I will let the user configure these display options to their liking and only render the things they want to be rendered.

Validation will also be prevalent because the matrix expressions will need to follow a strict format, which will be validated. The buttons to render and animate the matrix will only be clickable when the given expression is valid, so I will need to check this and update the buttons every time the text in the text box is changed. I will also need to parse matrix expressions so that I can evaluate them properly. All this validation ensures that crashes due to malformed input are practically impossible, and makes the user's life easier since they don't need to worry about if their input is in the right format - the app will tell them.

I will also make use of iteration, primarily in animation. I will have to re-calculate positions and values to render everything for every frame of the animation and this will likely be done with a simple \texttt{for} loop. A \texttt{for} loop will allow me to just loop over every frame and use the counter variable as a way to measure how far through the animation we are on each frame. This is preferable to a \texttt{while} loop, since that would require me to keep track of which frame we're on with a separate variable.

Finally, the core of the application is visualization, so that will definitely be used a lot. I will have to calculate positions of points and lines based on given matrices, and when animating, I will also have to calculate these matrices based on the current frame. Then I will have to use the rendering capabilities of the GUI framework that I choose to render these calculated points and lines onto a widget, which will form the viewport of the main GUI. I may also have to convert between coordinate systems. I will have the origin in the middle with positive $x$ going to the right and positive $y$ going up, but I may need to convert that to standard computer graphics coordinates with the origin in the top left, positive $x$ going to the right, and positive $y$ going down. This visualization of linear transformations is the core component of the app and is the primary feature, so it is incredibly important.

\subsection{Stakeholders}

Here's the section on stakeholders.

\end{document}
