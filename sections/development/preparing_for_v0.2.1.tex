% lintrans - The linear transformation visualizer
% Copyright (C) 2021-2022 D. Dyson (DoctorDalek1963)

% This program is licensed under GNU GPLv3, available here:
% <https://www.gnu.org/licenses/gpl-3.0.html>

\documentclass[../development.tex]{subfiles}

\begin{document}

\subsubsection{Fixing slots and signals\label{development:preparing-for-v0.2.1:fixing-slots-and-signals}}

I was perusing the Qt5 documentation when I learned about the difference between the \pyinline{dialog.finished} signal and the \pyinline{dialog.accepted} signal. I decided to rework some old code to make better use these signals.

When defining a new matrix or dialog settings, we only want to save the new data if the user actually accepted the dialog by clicking the confirm button. We don't want to save it if they clicked cancel. However, in the case of the error message dialog, we always want to update the render buttons when it's closed, no matter how the user closed the dialog.

%: 66242465222a153a5f37c4a1a3c2bd50bfd90933
%: src/lintrans/gui/main_window.py:35,447-448,461-469,478-483,493,511-512 noscopes

I also added the \pyinline{@pyqtSlot()} decorator to all the relevant methods in the matrix definition dialogs. The types in the brackets indicate the signature of the method.

A slot in Qt5 is just a method that is expected to be connected to a signal, so it gets called from the event loop. Using the decorator makes it clear that a method is a slot, and also allows slightly better performance.

%: 9beff9cf25d3af655e134205572a5668279f42cc
%: src/lintrans/gui/dialogs/define_new_matrix.py:67,138-140,143-144,155-157,164,194-195,203-204,214-215,226,276-277,288-289,305-306,317,348-349,356-357,364-365 noscopes

\subsubsection{Linking in documentation\label{development:preparing-for-v0.2.1:linking-in-documentation}}

I've been using Sphinx\cite{sphinx} for my documentation this whole time, and I've been using the Sphinx extension \texttt{intersphinx} to link to the Python standard library documentation. It uses a system of binary inventory files which define a reference map between names to use in the documentation, and where to link those names to. I recently learned of the \texttt{sphobjinv} Python package, which allows you to easily create your own local inventory files to reference some external source of documentation, such as the Qt5 documentation. I read through the \texttt{sphobjinv} documentation\cite{sphobjinv-2.2-docs} and designed a small script to read a custom text file, and create the binary inventory file needed by Sphinx.

%: 5455265a51666e29ab976152c1a758a422e1004a
%: docs/create_objects_inv.py:9-64

Line 11 should say \enquote{must have .html suffices}.

%: aab8e88b0e2cdae8038c9935031c74bcdae0ad5c
%: docs/pyqt5-objects.txt language="lexers.py:SphObjInvTextLexer -x"

I then just had to change all the references to Qt5 things in the documentation and then Sphinx would automatically link all the Qt5 references to their appropriate links, as defined in this file.

\subsubsection{Improving tests\label{development:preparing-for-v0.2.1:improving-tests}}

I made some small improvements to the unit tests by making sure they handled greedy index parsing, which means that something like \texttt{A\textasciicircum2 3B} will get parsed as \texttt{A\textasciicircum\{23\}B} because whitespace is ignored, as well asserting that all invalid expressions raise \pyinline{MatrixParseError}. I also added the copyright comment to the test files.

This was a tiny change, but worth noting.

%: c07d97024e1fe00ab110f43e5c7e6737c955d680
%: tests/matrices/test_parse_and_validate_expression.py:41,50-51,55,72-73,75,78-83

%: 70e1a7271a61f3009cc4d342f46743b248498a1c
%: tests/matrices/test_parse_and_validate_expression.py:26-31,86-90

\subsubsection{The Windows version file\label{development:preparing-for-v0.2.1:the-windows-version-file}}

Windows stores metadata for \texttt{.exe} files inside the \texttt{.exe} files themselve\cite{msdocs-version-information-structures}\cite{msdocs-versioninfo-resource}. PyInstaller lets you embed that metadata when compiling on Windows, using a version file\cite{pyintaller-4.10-capturing-windows-version-data}. Obviously, I wanted to include this metadata in my compiled \texttt{.exe} in the release.

I created a simple \texttt{precompile.py} script that would create different pre-compilation artefacts. I could then use these in the compilation workflow for GitHub Actions. I started with just Windows, but obviously I'm going to expand into Linux and macOS as well later.

%: 2126959cb6f836b1bc6c92dad859b43cbd86e1ab
%: precompile.py:9-103

%: 2126959cb6f836b1bc6c92dad859b43cbd86e1ab
%: .github/workflows/compile-release.yaml:8,60,68,87-104 noscopes language=yaml

I quickly realised that this design would make the compilation process more complicated, since the process would be in separate parts. Instead, I decided to create a unified compilation script, which runs a pre-compile step dependent on the operating system, and then compiles the program with the \pyinline{PyInstaller.__main__.run()} function.

%: ca674c7f7d61e8eed3410d456787cfe5b2bc28e5
%: compile.py:9-180

This new compilation script captures the whole process. On Linux or macOS, it just compiles the program with PyInstaller. On Windows, it has to generate the version file, then compile the program with an additional argument to include the version file. I then updated the GitHub Actions workflow to use this new compilation script.

%: e47fe732954bf018128bdcb5ee9c354910517f36
%: .github/workflows/compile-release.yaml:8,60,68,87-88 noscopes language=yaml

\end{document}
