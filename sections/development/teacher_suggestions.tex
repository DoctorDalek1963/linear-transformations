% lintrans - The linear transformation visualizer
% Copyright (C) 2021-2022 D. Dyson (DoctorDalek1963)

% This program is licensed under GNU GPLv3, available here:
% <https://www.gnu.org/licenses/gpl-3.0.html>

\documentclass[../development.tex]{subfiles}

\begin{document}

Now that I've got \texttt{v0.2.2} working and released, it's time to talk to some teachers. Ms Arnold is going to be the one that's actually using it, so obviously I needed to talk to her about it, but she wasn't available for a few days. I was talking to another maths teacher, Mr Dunkley, about my personal statement and took the opportunity to show him \texttt{lintrans} as well.

He was able to download and run it on a native Windows 10 installation, which was quite a relief. I'd been unable to test it on a real bare-metal Windows installation so far; I only had virtual machines.

He suggested two main improvements: \begin{enumerate}
	\item In the visual definition dialog, the basis vectors should snap to integer coordinates when they get close enough
	\item There should be a dialog box which displays matrices that you've already defined
\end{enumerate}

\subsubsection{Fixing a bug with animating stretches\label{development:teacher-suggestions:fixing-a-bug-with-animating-stretches}}

While Mr Dunkley was playing around with \texttt{lintrans}, I noticed a bug where animating uniform stretches, where every direction gets stretched equally work, just fine, but a matrix like $\begin{pmatrix}3 & 0\\ 0 & 1\end{pmatrix}$ would be animated as if it was $\begin{pmatrix}3 & 0\\ 0 & 3\end{pmatrix}$ and the plane would stretch equally in all directions.

After looking at the code later, I realised this was a problem with my new animation code detecting this type of matrix as a rotation. To fix this, I could just add a check to the \texttt{if} statement before the rotation animation code. This new check would make sure that the basis vectors of the application matrix were approximately the same length. By checking this as well, I fixed the bug.

%: eb118f02db9fb9c76ce15a976f81a037415d2a4e
%: src/lintrans/gui/main_window.py:410-417 highlight=417

\subsubsection{Fixing a hang after closing during an animation\label{development:teacher-suggestions:fixing-a-hang-after-closing-during-an-animation}}

This bug was tiny and only really affected me, but when I run \texttt{lintrans} from the terminal with \mintinline{sh}{python -m lintrans} and close it during an animation, I have to wait a moment to get my shell prompt back. Clearly, the program is still running even after the main window is closed.

I'm not entirely sure what was happening here, but it was something to do with Qt5's event loop and threading model, and I could fix the bug by just overriding \pyinline{LintransMainWindow.closeEvent()} to set \pyinline{self.animating = False} and that would ensure that animations are stopped before the main window closes.

%: d60c1878058799be544f5cf0847b478dccd3a021
%: src/lintrans/gui/main_window.py:262-265

\subsubsection{Adding snapping in the visual definition dialog\label{development:teacher-suggestions:adding-snapping-in-the-visual-definition-dialog}}

I was now ready to implement snapping in the visual definition dialog.

The \pyinline{DefineVisuallyWidget} class has a \pyinline{self.dragged_point} attribute, which is the coordinates of the point being dragged. In \pyinline{mouseMoveEvent()}, we get the coordinates of the current cursor position and set the dragged point accordingly. That's all that happens in the old code.

%: d60c1878058799be544f5cf0847b478dccd3a021
%: src/lintrans/gui/plots/widgets.py:127-147

To add snapping to this, we can just get the 4 integer coordinates around the dragged point, check each of their distances to the dragged point, and if it's sufficiently close to one of them, snap it to that point.

%: f2de39fec299bb8ed155ee649d34c9063787e71a
%: src/lintrans/gui/plots/widgets.py:128-165

This all worked very well.

\subsubsection{Respecting transitional animation in animation sequences\label{development:teacher-suggestions:respecting-transitional-animation-in-animation-sequences}}

It's good to animate a sequence of matrices by applying them one after another, but it can also be beneficial to animate a sequence by just moving between them in a transitional animation style. This is already possible for individual animations by disabling the \textit{Applicative animation} display setting, but it would be nice to respect this setting for animation sequences.

To do this, I just have to add a few lines to the part where we do animation sequences.

%: 41907b81661f3878e435b794d9d719491ef14237
%: src/lintrans/gui/main_window.py:339-367 highlight=348-351

\end{document}
