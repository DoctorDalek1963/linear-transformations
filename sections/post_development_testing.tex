% lintrans - The linear transformation visualizer
% Copyright (C) 2021-2022 D. Dyson (DoctorDalek1963)

% This program is licensed under GNU GPLv3, available here:
% <https://www.gnu.org/licenses/gpl-3.0.html>

\documentclass[../main.tex]{subfiles}

\begin{document}

% Figures are x.y
\numberwithin{figure}{section}

\begin{landscape}
\section{Post-development Testing\label{evaluation}}

\begin{longtable}[c]{|m{1.4cm}||p{3cm}|p{2.5cm}|p{3.5cm}|p{2.9cm}|p{2.5cm}|p{5cm}|}
	\hline
	\textbf{Test ID}
		& \textbf{Description}
		& \textbf{Test data}
		& \textbf{Expected outcome}
		& \textbf{Actual outcome}
		& \textbf{Evidence}
		& \textbf{Comments}
	\\ \hline
	\testnum{compile}
		& The program should run in the Python interpreter and compile with PyInstaller.
		& N/A
		& The program runs and compiles with no problem.
		& As expected.
		& See Figure~\ref{fig:pdtest:compile.png}.
		& None.
	\\ \hline
	\testnum{define-visually}
		& The user should be able to define matrices visually.
		& Approximately $\begin{pmatrix}2 & 0.5\\ 1 & 2\end{pmatrix}$
		& The matrix should be successfully defined by the visual definition dialog.
		& The definition works fine with no errors and the defined matrix appears in the info panel.
		& See Figures~\ref{fig:pdtest:define-visually.png} and \ref{fig:pdtest:info-panel-no-expression.png}
		& It's hard to correctly line up the $j$ vector since it's $\begin{pmatrix}0.5\\ 2\end{pmatrix}$
			and the dialog only snaps to integer coordinates, but defining $i$ was easy because of this.
	\\ \hline
	\testnum{define-numerically}
		& The user should be able to define matrices numerically.
		& $\begin{pmatrix}2.2 & 18\\ -0.4529 & 0\end{pmatrix}$
		& The matrix should be successfully defined by the numerical definition dialog.
		& The definition works fine with no errors and the defined matrix appears in the info panel.
		& See Figures~\ref{fig:pdtest:define-numerically.png} and \ref{fig:pdtest:info-panel-no-expression.png}
		& Inputting the numbers directly makes it easy to get exactly what you want (such as precise decimals and
			large numbers that can be inconvenient to zoom out to find in the visual definition dialog) but there's
			no support for mathematical expressions or fractions, so the user is limited to decimal numbers.
	\\ \hline
	\testnum{render-defined}
		& The program should be able to render matrices that the user has defined.
		& Rendering the previously defined $\mathbf{A}$ and then $\mathbf{B}$.
		& The program renders the matrices correctly.
		& As expected.
		& See Figures~\ref{fig:pdtest:render-defined-A.png} and \ref{fig:pdtest:render-defined-B.png}.
		& Both the button and hotkey work as expected, as does the reset button.
	\\ \hline
	\testnum{animate-defined}
		& The program should be able to animate matrices that the user has defined.
		& Resetting, then animating $\mathbf{A}$, then resetting and animating $\mathbf{B}$>
		& The program animates smoothly from I to the matrices without any issues.
		& As expected.
		& See \texttt{pdtest/animate_defined.mp4} in the videos folder.
		& This test was performed with the default display settings.
	\\ \hline
	\testnum{render-and-animate-rotations}
		& The program should be able to render and animate arbitrary rotations created with the \texttt{rot()} command.
		& Trying to render and animate \texttt{rot(135)}.
		& The program renders and animates the rotations correctly.
		& As expected.
		& See \texttt{pdtest/render_and_animate_rotations.mp4} in the videos folder.
		& None.
	\\ \hline
	\testnum{render-and-animate-expressions}
		& The program should be able to render and animate expressions containing previously defined matrices.
		& Trying to render and animate $\mathbf{AB}$ and $\mathbf{A}^{-1}\mathbf{B}^2$.
		& The program renders and animates the expressions correctly.
		& As expected.
		& See \texttt{pdtest/render_and_animate_expressions.mp4} in the videos folder.
		& None.
	\\ \hline
	\testnum{animate-multiple-transitional}
		& The program should be able to animate one matrix and then animate a transition to another.
		& Trying to render and animate $\mathbf{A}$ and then $\mathbf{B}$.
		& The program animates them correctly, transitioning between them.
		& As expected.
		& See \texttt{pdtest/animate_multiple_transitional.mp4} in the videos folder.
		& None.
	\\ \hline
	\testnum{animate-multiple-applicative}
		& The program should be able to animate one matrix and then apply another matrix to it.
		& I will start at $\mathbf{I}$, then animate $\mathbf{B}$, then apply $\mathbf{A}$. This should be
		equivalent to $\mathbf{AB}$, so I will apply $(\mathbf{AB})^{-1}$ to undo it.
		& The program animates everything correctly.
		& As expected.
		& See \texttt{pdtest/animate_multiple_applicative.mp4} in the videos folder.
		& None.
	\\ \hline
\end{longtable}
\end{landscape}

\simplefig{pdtest/compile.png}{The successful output of compilation}
\simplefig{pdtest/define-visually.png}{The matrix $\mathbf{A}$ being defined visually}
\simplefig[0.8]{pdtest/define-numerically.png}{The matrix $\mathbf{B}$ being defined numerically}
\simplefig[0.6]{pdtest/info-panel-no-expression.png}{The info panel after defining matrices $\mathbf{A}$ and $\mathbf{B}$}
\simplefig{pdtest/render-defined-A.png}{The matrix $\mathbf{A}$ being rendered}
\simplefig{pdtest/render-defined-B.png}{The matrix $\mathbf{B}$ being rendered}

%animate sequence (transitional)
%animate sequence (applicative)
%define as expression
%change dependencies of expression
%determinant parallelogram
%eigenlines

% Figures are x.y.z
\numberwithin{figure}{subsection}

\end{document}
